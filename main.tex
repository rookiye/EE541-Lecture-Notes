\documentclass[a4paper,12pt]{article}
\usepackage[utf8]{inputenc}
\usepackage{geometry}
\usepackage{hyperref}
\usepackage{tocloft}
\usepackage{xcolor}

\geometry{left=1in, right=1in, top=1in, bottom=1in}

% 设置标题样式
\title{\textbf{\Huge Lecture Note of EE: 541 \\ A Computational Introduction to Deep Learning}}

\author{\vspace{1cm} \Large Dr. Brandon Franzke \\[1cm] \normalsize \textit{Scribe: Yi FAN}}
\date{}

% 自定义目录样式
\renewcommand{\cftsecfont}{\normalfont\color{blue}} % 二级标题字体蓝色
\renewcommand{\cftsubsecfont}{\normalfont} % 三级标题默认字体
\renewcommand{\cftsecpagefont}{\normalfont} % 页码默认字体
\setlength{\cftbeforesecskip}{0.5em} % 标题间距

\begin{document}

% 封面页
\begin{titlepage}
    \centering
    \vspace*{2cm} % 顶部空白
    \maketitle % 渲染标题
    \vfill % 自动填充剩余空间
\end{titlepage}

% 目录页
\tableofcontents
\newpage

% Main I
\clearpage
\begin{titlepage}
    \centering
    \vfill % 垂直方向居中
    {\Huge \textbf{Part I}}\\[1.5cm] % 第一行大标题
    {\LARGE \textbf{Basic theory}} % 第二行稍小的标题
    \vfill % 垂直方向居中
\end{titlepage}
\addcontentsline{toc}{section}{I \quad Basic theory}
\section{Introduction to EE 541: A Computational Introduction to Deep Learning}
Brief Intro
\subsection{What is Machine Learning}
what is 
\subsection{What is Deep Learning}
what is
\subsection{The history and development of Machine Learning}
History
\subsection{Your Python Environment}
Python
\subsection{Reference and Recommended Reading}
[1]xxx

% not a paranomic guidance to Python
% just clarify what Python skills will be used in this course

\clearpage
\section{Python Fundamentals}
Python
\subsection{Data types and control structures}
Data types and control structures
\subsection{Data Structures}
Data Structures
\subsection{Objects and classes}
Objects and classes
\subsection{Python Libraries: NumPy and Matplotlib}
Python Libraries: NumPy and Matplotlib
\subsection{Reference and Recommended Reading}
Reference and Recommended Reading

\clearpage
\section{Minimum Mean Square Error(MMSE) Estimation}
\subsection{Random vectors and Covariance Matrix}
Random vectors and Covariance Matrix
\subsection{What is Estimation and what is LMMSE}
What is Estimation and what is LMMSE
\subsection{MMSE Estimation for Jointly Gaussian case}
MMSE Estimation for Jointly Gaussian case
\subsection{Least Mean Square(LMS) Algorithm}
Least Mean Square(LMS) Algorithm
\subsection{Reference and Recommended Reading}
Reference and Recommended Reading

\clearpage
\section{Regression and Classification}
Regression and Classification
\subsection{Linear Regression problem}
Linear Regression problem
\subsection{Linear Least Squares Estimation Regression}
Linear Least Squares Estimation Regression
\subsection{Linear Classification problem}
Linear Classification problem
\subsection{Comparison between Regression and Classification}
Comparison between Regression and Classification
\subsection{Reference and Recommended Reading}

\clearpage
\section{Decision and Logistic Regression}
Decision and Logistic Regression
\subsection{Bayesian Decision Theory}
Bayesian Decision Theory
\subsection{Logistic Regression}
Logistic Regression
\subsection{Regularization}
Regularization
\subsection{Reference and Recommended Reading}

\clearpage
\section{Multilayer Perceptron networks (MLPS)}
Multilayer Perceptron networks (MLPS)
\subsection{MLP forward propagation}
MLP forward propagation
\subsection{Backward propagation}
Backward propagation
\subsection{LMS versus BP}
LMS versus BP
\subsection{Reference and Recommended Reading}

\clearpage
\section{Training Deep Neutral Networks}
Training Deep Neutral Networks
\subsection{Universal Approximation Theorem: Why Go Deep}
Universal Approximation Theorem: Why Go Deep
\subsection{Introduction to PyTorch}
Introduction to PyTorch
\subsection{Dealing with Data}
Dealing with Data
\subsection{Vanishing Gradients}
Vanishing Gradients
\subsection{Parameter Initialization}
Parameter Initialization
\subsection{Cost Function and Regularization}
Cost Function and Regularization
\subsection{Deep Learning Optimizers}
Deep Learning Optimizers
\subsection{Reference and Recommended Reading}

\clearpage
\section{Convolutional Neutral Networks(CNNs)}
Convolutional Neutral Networks(CNNs)
\subsection{Convolution operations}
Convolution operations
\subsection{Conv2D Layer in PyTorch}
Conv2D Layer in PyTorch
\subsection{Pooling and Stride}
Pooling and Stride
\subsection{Visualization}
Visualization
\subsection{Block Structures}
Block Structures
\subsection{Open source deep learning framework: Keras}
\href{https://keras.io/api/applications/}{Keras Applications API}
\subsection{Reference and Recommended Reading}

\clearpage
\section{Working with Data}
Working with Data
\subsection{Designing Datasets}
Designing Datasets
\subsection{Deep Learning workflow}
Deep Learning workflow
\subsection{Normalization Methods}
Normalization Methods
\subsection{Dimensionality Reduction: PCA and LDA}
Dimensionality Reduction: PCA and LDA
\subsection{Finding data}
Finding data
\subsection{Reference and Recommended Reading}
%  II

\clearpage
\begin{titlepage}
    \centering
    \vfill 
    {\Huge \textbf{Part II}}\\[1.5cm] ]
    {\LARGE \textbf{Application of Deep Learning}} ]
    \vfill 
\end{titlepage}
\addcontentsline{toc}{section}{II \quad Application of Deep Learning}

\section{Adapt pre-trained models to new datasets }
About Homework 7 

\end{document}
